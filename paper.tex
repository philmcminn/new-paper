% The title of the paper
\newcommand{\papertitle}{A New Paper}

% The authors of the paper.
% Usually different conference formats have different 
% ways of including this information, so this command
% is not needed.
\newcommand{\paperauthors}{
    Phil McMinn
}

% Define document class and default bibliography style
\documentclass[12pt]{article}
\usepackage[margin=3cm]{geometry}

\newcommand{\paperbibliographystyle}{plain}

% Pre-abstract formatting
\newcommand{\preabstract}{
    \title{\papertitle}

    % author block
    \author{
        \ifcsname doubleblind\endcsname
            \doubleblind
        \else
            \firstauthor \\ \firstauthoraffiliation 
            \ifcsname collaboratortwo\endcsname
                \and \collaboratortwo \\ \collaboratortwoaffiliation
            \fi
            \ifcsname collaboratortwo\endcsname
                \and \collaboratorone \\ \collaboratoroneaffiliation
            \fi
            \ifcsname collaboratorthree\endcsname
                \and \collaboratorthree \\ \collaboratorthreeaffiliation 
            \fi
            \ifcsname lastauthor\endcsname
                \and \lastauthor \\ \lastauthoraffiliation
            \fi
        \fi
    }

    % Write out the date?
    \ifcsname writedate\endcsname%
        \date{}
    \fi%

    \maketitle
}

% Post-abstract formatting
\newcommand{\postabstract}{
}

% Packages used by the paper

\usepackage{booktabs}
\usepackage{xspace}
% Custom commands used by the paper

% Suppress details if double-blind, else write them out
\newcommand{\ifnotdoubleblind}[1] {
    \ifcsname doubleblind\endcsname
        \doubleblind
    \else 
      #1 
    \fi
}
% Constants that appear throughout the paper

% e.g.
% \newcommand{\numsubjects}{34}
% Words and phrases that need to be spelt
% and formatted consistently

\newcommand{\etc}{etc.\xspace}
\newcommand{\etal}{et~al.\xspace}
\newcommand{\ie}{i.e.\xspace}

\begin{document}

% The \papertitle command lives in details.tex
% It is useful to have it defined as a command so that
% it can be referenced in other documents,
% e.g. a referee response letter
\title{\papertitle}

% The \paperauthors command lives in details.tex
% A publisher's template may have different
% commands for adding the authors to a paper
\author{\paperauthors}

% This is usually not needed:
\date{}

\maketitle

% Sections of the document. 
% These are some typical ones.
% Add more if you need them.
\begin{abstract}
    The abstract of the paper goes here.
        
    End with some figures that give hard evidence
    of the benefit of your approach.
    
    Note that some conferences and journals
    limit the number of words in an abstract to
    200 or 250.
\end{abstract}
\section{Introduction}
\label{sec:introduction}

This section sets the stage for the paper. 
It should clearly explain the problem addressed by the 
paper, and give high level details of your approach
for tackling it, explaining why no other paper already
addresses the exact same problem.

Right before this section concludes, it should always
feature a list of contributions that the paper makes:

\begin{enumerate}

    \item Contribution 1. Description.

    \item Contribution 2. Description.

\end{enumerate}    

Try to avoid concluding the introduction with a 
``This paper is organized as follows\ldots'' paragraph,
as they are rarely insightful! 
%
However, you may want to include a sentence or two 
leading into the next section --- Background.
\section{Background}
\label{sec:background}

The background section introduces past work (by you
and others) that the reader needs to know to understand 
the rest of your paper. 
%
It is different from the ``Related Work'' section, 
% Always reference sections like this, i.e. with
% a capital letter on "Section", followed by a 
% non-breaking space (the ~ character), which
% keeps the word "Section" and the following
% number on teh same line.
Section~\ref{sec:related-work},
in which you should cite work related to your paper,
but which is not integral to the basis of your approach.

% You can copy this file for every new figure.
%
% Figures should be placed at the top of pages/columns
% where they can.
%
% This can be ensured by using the [t] parameter to the
% "\begin{figure}" declaration.
%
\begin{figure}[t]
    % Figures should be centered in the page/column
    \centering

    % Figure content goes here.
    \includegraphics[width=\columnwidth]{graphics/figure.pdf}

    \caption{
        % The label should appear _inside_ the caption to ensure
        % Latex numbers it correctly. This is a common gotcha!
        %
        % All figure labels should start with "fig:"
        % So that the figure file can be found easily, the rest of the
        % figure label should be the same as the filename, as it is
        % in this example:
        %
        \label{fig:plain-figure}
        %
        Add your caption here. Captions for figures go {\em below}
        the figure.
    }
\end{figure}
% Tables should be placed at the top of pages/columns
% where they can.
%
% This can be ensured by using the [t] parameter to the
% "\begin{table}" declaration.
%
\begin{table}[t]
    % Figures should be centered in the page/column
    \centering
    %
    \caption{
        % The label should appear _inside_ the caption to ensure
        % Latex numbers it correctly. This is a common gotcha!
        %
        % All table labels should start with "tab:"
        % So that the figure file can be found easily, the rest of the
        % table's label should be the same as the filename, as it is
        % in this example:
        %
        \label{tab:plain-table}
        %
        Add your caption here. Captions for tables go {\em above}
        the table.
    }
    %
    % Depending on the template, some breathing space might need to
    % be added (use \smallskip \medskip etc.) here
    %
    % Or, to save space, you might want to remove space
    % (use a negative \vspace, e.g. \vspace{-1em})
    %
    %
    % Table content goes here. Use this file to specify the
    % table's column headings. The data should be automatically
    % output from a program processing the raw experimental data
    % and should be inputted from another file. This enables
    % the data to change, if for example, the experiment data
    % needs to be updated.
    %
    % Do not use vertical rules. Ensure you use \toprule, \midrule
    % and \bottomrule from the "booktabs" package effectively.
    %
    % Numbers should be right justified (use "r"), 
    % text left justified (use "l").
    %
    % For example:
    %
    \begin{tabular}{lr}
        \toprule
            % use a new line for each column if needed
            {\bf Algorithm}  &
            {\bf Best Fitness}
            \\

        \midrule

        % The data in this file should be automatically output
% from code that processes raw results data.
DOG         & 78  \\
CAT         & 60  \\

\midrule

{\bf Mean}  & 69  \\
{\bf Total} & 138 \\

        \bottomrule
    \end{tabular}
    %
    % To save space, you might want to remove space here
    % (use a negative \vspace, e.g. \vspace{-1em})
\end{table}


\section{Approach}
\label{sec:approach}

This section should detail your approach as
clearly as possible, at a high enough level
that the reader can understand. This section
may include a motivating example, explaining
why your approach is needed.

You may want to include some algorithms and
figures to help illustrate the points you
are making, to introduce the motivating
example, or give an overview of the architecture
of your tool.

The information in {\tt figures/plain-figure.tex}
describes how to format a figure, included here as
% Always reference figures like you would reference
% sections as demonstrated in the sections/background.tex
Figure~\ref{fig:plain-figure}.

\section{Evaluation}
\label{sec:evaluation}

% Your text goes here
\section{Results}
\label{sec:results}

In this section, you should relate your results
to each of your research questions, one by one.

Any more anecdotal observations, or anything you
observed in the course of answering your research
questions that you did not deliberately set out 
to investigate should be included
in a subsection at the end of this section, called
``Discussion''.

\section{Related Work}
\label{sec:related-work}

This section should cite any work related to your paper, but which is not
integral to the basis of your approach.

It is a useful catch-all for anything that did not appear in background, and
helps satisfy the referees that you know your field by demonstrating a knowledge
of other work that is going on in the area.

While discussing related work, it is important to keep making it clear to the
reader the ways in which each paper is different from your own, and/or addresses
a different problem. In general, it is a mistake to be overly critical of
others' work, unless discussing the limitations of that work helps to draw out
key differences between their approach and yours.

\section{Conclusions and Future Work}
\label{sec:conclusions-and-future-work}

This section should begin by reminded the reader
about your approach and its motivation, and that
no other previous research addresses the same
problem.

You may want to include some more summary statistics
from the results, demonstrating its worth.

You should then conclude the paper with some ideas
for future work.

% The \paperbibliographystyle command is defined
% by the paper's style file ... see the file
% referenced by the \input{styles/...}
% declaration at the top of this file
\bibliographystyle{\paperbibliographystyle}

% The bibliography should be a Git subproject
% imported into this repository
%
% \bibliography{...}

\end{document}